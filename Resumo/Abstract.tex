\begin{resumo}[Abstract]
 \begin{otherlanguage*}{english}
  The Hardware in the Loop (HIL) technique is fundamental for real-time simulations in order to connect
  a part of the system to a digital model. The technique consists in the insertion of a physical device in
  the control loop during the system's development. Currently, HIL simulations are mainly used for the 
  development of new components and actuators in several fields such as: flight simulations, 
  electronic power systems, mobile robotics and traffic engineering. For the system studied in this monography, the HIL
  technique is applied to a simplified robotic manipulator with three revolute joints. The method was
  applied in three distinct steps: the first one consists in surveying the plant model and tuning the
  controller in a simulation environment; the second is the replacement of the virtual controller by
  its implementation in a physical device followed by the validation within the simulation environment;
  finally, the last step corresponds to the use of the controller implemented in the physical device
  communicating directly with the actual plant. The project makes use of the single board computer 
  Raspberry Pi as the hardware inserted into the loop of the manipulator's control. Finally, an independent
  joint control for the manipulator applying, applying the HIL technique.

  \vspace{\onelineskip}
 
   \noindent 
   \textbf{Keywords}: Hardware in the Loop, robotic manipulator, control, independent joints.
 \end{otherlanguage*}
\end{resumo}