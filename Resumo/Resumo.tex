\setlength{\absparsep}{18pt} % ajusta o espaçamento dos parágrafos do resumo
\begin{resumo}
  A técnica \textit{Hardware in the Loop} (HIL) é fundamental para simulações em tempo real a fim de conectar uma parte 
  do sistema a um modelo digital. A técnica consiste na inserção de um dispositivo físico na malha de controle
  durante o desenvolvimento do sistema. Atualmente, as simulações HIL são utilizadas principalmente para o 
  desenvolvimento de novos componentes e atuadores em vários campos diferentes como: simulações de vôo, 
  sistemas eletrônicos de potência, robótica móvel e engenharia de tráfego. No sistema desenvolvido, a técnica 
  HIL é aplicada a um manipulador robótico simplificado para três juntas de revolução. O método foi aplicado em 
  três etapas distintas. A primeira consiste no levantamento do modelo da planta e sintonia do controlador em 
  ambiente de simulação. A segunda na substituição do controlador virtual por sua implementação em dispositivo 
  físico seguido da validação em conjunto com o ambiente de simulação. Por fim, a última etapa corresponde na 
  utilização do controlador implementado no dispositivo físico comunicando diretamente com a planta real. O projeto
  faz uso do computador de placa única Raspberry como o Hardware inserido na malha de controle do manipulador.
  Finalmente, destaca-se como objetivo do projeto a sintonia e validação do controle de juntas independentes do
  manipulador em questão aplicando a técnica HIL.
  
 \textbf{Palavras-chave}: \textit{Hardware in the Loop}. manipulador robótico. controle. juntas independentes.
\end{resumo}