\chapter{Conclusão}
%\markboth{\thechapter ~~~ Conclusão}{}
%\label{revBib}

O controle de trajetória de uma manipulador robótico visa configurar o comportamento das juntas de tal 
forma que a ferramenta de trabalho siga uma trajeto de referência. Existem algumas formas de se modelar
o comportamento dessas juntas como o modelo \textit{Denavit-Hartenberg} e as equações de \textit{Euler-Lagrange},
além do ensaio em malha aberta para obtenção do modelo. No presente projeto, a modelagem foi tratada das duas formas
e o projeto de controle foi realizado pelo método do lugar das raízes. A sintonia e validação da estratégia de controle
foi realizada por meio do uso da técnica HIL.

A técnica HIL, que consiste em inserir um dispositivo físico na malha de controle simulada, foi dividida
em três etapas neste projeto. O resultado obtido em cada uma das etapas foi exposto através
dos gráficos com o objetivo de que o padrão se repetisse entre as etapas. Além disso, a sintonia
e validação da estratégia de controle para a planta em questão é fundamental para o êxito da 
conclusão do projeto.

Analisando o exposto no capítulo anterior, observa-se que as considerações feitas para simplificar
a modelagem \textit{Denavit-Hartenberg} e as equações de \textit{Euler-Lagrange} tornaram o desenvolvimento
da solução para o modelo da planta fisicamente inviável. A solução mais coerente encontrada foi o ensaio 
em malha aberta para a obtenção do modelo de cada junta do manipulador, seguido do controle de junta independente.

Os experimentos demonstraram que a modelagem do manipulador teve um resultado coerente com a planta real, sendo que as curvas
de resposta ao degrau se sobrepuseram e os parâmetros de tempo de acomodação e máximo sobressinal ficaram bastante similares como desejado. 
O projeto de controle pelo método do lugar das raízes teve como resultado um controlador que tornou a dinâmica da 
planta mais lenta. Por outro lado, o controlador PI fez o sistema ficar mais robusto, rejeitando distúrbios ao fechar
a malha além de tornar a transição dos estados mais suave (é possível notar a diferença entre as respostas apresentadas na 
\autoref{fig:ensaioMalhaAberta} e na \autoref{fig:hilFase3}). Além disso, as respostas ao degrau dos modelos obtidos
foram coerentes e apresentaram o mesmo comportamento em todas as etapas do método HIL. Assim, a sintonia e
validação da estratégia de controle foi realizada com êxito.

Por fim, o desafio futuro envolve tanto a planta do manipulador quanto a configuração \textit{Hardware in the loop}. Para o
manipulador é necessário reduzir ou remover a trepidação observada principalmente na base. Essa trepidação foi reduzida ao
validar a estratégia de controle (conforme a \autoref{fig:hilFase3} mostra), mas ela ainda perdura.
A implementação de outras estratégias de controle pode auxiliar na redução da trepidação das juntas e aumento da robustez do sistema.
Em relação à técnica HIL, é de grande importância optar por um \textit{hardware} de tempo real, o que não foi atendido com a \textit{Raspberry Pi}.
A solução mais coerente seria modificar o \textit{kernel} \textit{Linux} da \textit{Raspberry Pi} para tornar o tempo 
de resposta aos eventos um tempo pré-definido.



