\chapter{Conclusão}
%\markboth{\thechapter ~~~ Conclusão}{}
%\label{revBib}

A técnica HIL, que consiste em inserir um dispositivo físico na malha de controle, foi dividida
em três etapas neste projeto. O resultado obtido em cada uma das etapas foi exposto através
dos gráficos, e o objetivo é que o padrão se repita entre as etapas. Além disso, a sintonia
e validação da estratégia de controle para a planta em questão é fundamental para o êxito da 
conclusão do projeto.

Analisando o exposto no capítulo anterior, observa-se que as considerações feitas para simplificar
a modelagem \textit{Denavit-Hartenberg} e as equações de \textit{Euler-Lagrange} tornaram o desenvolvimento
da solução para o modelo da planta fisicamente inviável. A solução mais coerente encontrada foi o ensaio 
em malha aberta para a obtenção do modelo de cada junta do manipulador, seguido do controle de junta independente.

A \autoref{fig:ensaioMalhaAbertaSimul} evidencia que a modelagem teve um resultado bastante coerente com a planta real, 
as curvas se sobrepuseram e os parâmetros como o tempo de acomodação e o sobressinal coincidiram com o esperado. 
O projeto de controle pelo método do lugar das raízes teve como resultado um controlador que tornou a dinâmica da 
planta mais lenta. Por outro lado, o controlador PI fez o sistema ficar mais robusto, rejeitando distúrbios ao fechar
a malha além de tornar a transição dos estados mais suave (é possível notar a diferença entre a 
\autoref{fig:ensaioMalhaAberta} e a \autoref{fig:hilFase3}). Além disso, o padrão da resposta do sistema ao longo do 
tempo se repetiu entre as etapas do método HIL. Sendo assim, o conjunto dos tópicos citados acima resultou na sintonia e
validação da estratégia de controle.

Por fim, o desafio futuro envolve tanto a planta do manipulador quanto a configuração \textit{Hardware in the loop}. Para o
manipulador é necessário reduzir ou remover a trepidação observada principalmente na base. Essa trepidação foi reduzida ao
validar a estratégia de controle (conforme a \autoref{fig:hilFase3} mostra), mas ela ainda ocorre. Em relação à técnica HIL,
é de grande importância optar por um \textit{hardware} de tempo real, o que não foi atendido com a \textit{Raspberry}. A 
solução atual mais coerente seria modificar o \textit{kernel} \textit{Linux} da \textit{Raspberry} para tornar o tempo 
de resposta aos eventos um tempo pré-definido.



