\chapter{Revisão Bibliográfica}
%\markboth{\thechapter ~~~ Revisão Bibliográfica}{}
%\label{revBib}

Esta monografia pretende validar estratégias de controle em um manipulador robótico 
através da técnica \textit{Hardware in the Loop} (HIL). Por este motivo, este capítulo 
aborda uma breve revisão dos principais aspectos que envolvem o tema proposto: manipuladores 
robóticos, sistemas de controle e HIL. Por clareza, esses tópicos são apresentados em seções 
distintas.


\section{Manipuladores Robóticos}
\markright{\thesection ~~~ Manipuladores Robóticos}
%\label{manipRob}

Um manipulador robótico pode ser definido como um mecanismo reprogramável e multifuncional
que é desenvolvido para mover materiais, peças e ferramentas \cite{Murphy:2000:IAR:517685}. 
Mecanismo este que é composto por elos e juntas mecânicas. Apesar disso, o manipulador não pode ser 
visto apenas como uma série de elos (ou \textit{links}) em cascata. Para \citeonline{Spong}, 
o manipulador robótico é composto por um braço mecânico, pela ferramenta no fim do braço, 
pela fonte de energia externa, pelos sensores externos e internos, pela interface de comunicação
com o sistema e pelo controle do microcontrolador.

Na composição do manipulador, os elos são conectados pelas juntas formando a cadeia 
cinemática. Segundo \citeonline{paul1981robot}, ao incorporar coordenadas em cada elo do manipulador, 
usando transformação homogênea, é possível descrever a posição relativa e a orientação entre elas. 
As juntas podem ser de revolução ou prismáticas \cite{paul1981robot}. As juntas de revolução
são aquelas que permitem um movimento de rotação entre um elo e outro. Por outro lado,
as prismáticas são as que possibilitam apenas um movimento linear entre os elos.

A forma geométrica de se classificar os manipuladores é dada pela disposição das juntas 
na cadeia cinemática. Segundo \citeonline{Spong}, a maioria dos manipuladores se 
enquadradia em uma das categorias a seguir (em que R corresponde a uma junta de revolução 
e P uma junta prismática): articulada (RRR), esférica (RRP), SCARA (RRP), cilíndrica 
(RPP), ou Cartesiana (PPP).

O grau de liberdade (DOF - \textit{degrees-of-freedom}) é um parâmetro fundamental para 
a configuração espacial do manipulador robótico. É ele quem define qual a dimensão do 
espaço de configuração, ou seja, um manipulador tem \textit{n} graus de liberdade caso sua 
configuração seja minimamente especificada por \textit{n} parâmetros \cite{Spong}. Para 
\citeonline{Spong}, a maioria dos manipuladores industriais atualmente possuem seis 
ou menos graus de liberdade.

\section{Sistemas de Controle}
\markright{\thesection ~~~ Sistemas de Controle}
%\label{manipRob}

Segundo \citeonline{Phillips}, um controlador é necessário em uma planta 
para processar um sinal de erro de forma a atender certas especificações pré 
definidas. Esse sinal de erro é dado pela diferença entre a resposta do sistema, 
determinada por um sensor, e a trajetória desejada. Entre as especificações mais 
comuns em do controle de sistemas dinâmicos lineares estão: rejeição do 
distúrbio, erro em estado estacionário e a resposta transiente.

As variedades de controle se dão conforme os tipos de sinais existentes. Sinais 
analógicos são aqueles que apresentam valor em qualquer instante de tempo, os 
discretos apresentam valores em instantes múltiplos do tempo de amostragem 
\cite{Castrucci}. Por fim, os sinais digitais são os amostrados no tempo com uma 
amplitude representada em um número limitado de \textit{bits}, ou seja, a amplitude 
sofre o efeito da quantização. Segundo \citeonline{Castrucci}, o mais comum é
fazer o projeto do controlador analógico, e depois convertê-lo em digital para a 
execução computacional. Por outro lado, existem também formas de projetar o
controlador em algoritmos digitais diretamente.

Nos dias atuais, o controlador mais utilizado na indústria é o controlador PID. 
Segundo \citeonline{Ogata}, mais da metade dos controladores industriais empregam
o controle PID ou variantes dele. O seu sucesso está ligado diretamente a sua 
concepção robusta e sua aplicabilidade geral a maioria dos sistemas. O seu nome 
é referente a sua função de transferência (Equação \ref{eq:pid}), que é composta 
pelas ações proporcional, integradora e derivativa \cite{Castrucci}:

\begin{equation}
  \begin{gathered}
    u(t) = K_p\left(e(t)+\frac{1}{T_i}\int_{0}^{t}e(d\tau)d\tau+T_d\frac{de(t)}{dt}\right)
  \end{gathered}
  \label{eq:pid}
\end{equation}

Sendo:
\begin{itemize}
 \item \textit{u(t)}: sinal de saída do controlador, ou variável manipulada;
 \item \textit{e(t)}: sinal de entrada do controlador, ou erro entre a resposta do sistema e o sinal de referência;
 \item $K_p$, $T_i$, $T_d$: parâmetros de ajustes do PID.
\end{itemize}

\section{Técnica \textit{Hardware in the Loop (HIL)}}
\markright{\thesection ~~~ Hardware in the Loop}
%\label{hil}

A idéia básica da simulação de \textit{Hardware-in-the-Loop} (HIL) é a inclusão de 
uma parte do \textit{hardware} real no loop de simulação durante o desenvolvimento 
do sistema \cite{Bacic}. Ou seja, a técnica consiste em inserir um dispositivo físico 
na malha de controle de uma simulação. Nessa técnica, uma parte do sistema é integrada 
a uma outra parte que está sendo simulada em tempo real \cite{Abourida}.

Os primeiros usos da simulação \textit{Hardware-in-the-Loop} (HIL) estão relacionados com 
as simulações de voo \cite{Isermann}. Pouco tempo depois, a NASA desenvolveu simulações de alta 
fidelidade para o desenvolvimento de tecnologia de aeronaves altamente manobráveis \cite{Evans}. 
Outras aplicações dessas simulações vieram posteriormente com os testes dinâmicos de componentes 
de veículos, como, por exemplo, suspensão e corpo do carro \cite{Isermann}.

Para \citeonline{Abourida}, a técnica \textit{Hardware-in-the-Loop} (HIL) é fundamental
para simulações em tempo real, não para simular o sistema completo em tempo real, mas sim
para conectar uma parte do sistema a um modelo digital em tempo real. Além disso, essa técnica
de simulação tem como desafio o alcance da precisão de um modelo aceitável com um tempo de simulação 
digital viável \cite{Abourida}. Isso porque alguns sistemas (aqueles altamente não-lineares)
precisam de uma frequência de amostragem muito alta para alcançarem uma precisão aceitável.

\clearpage